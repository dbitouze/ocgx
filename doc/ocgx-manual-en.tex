\documentclass[a4paper]{ltxdoc}
\usepackage[vmargin=2cm,hmargin=3.5cm]{geometry}
\usepackage[latin1]{inputenc}
\usepackage[T1]{fontenc}
\usepackage{lmodern}
\usepackage{microtype}
\usepackage{xcolor}
\usepackage{ifthen}
\usepackage{doc}
\usepackage{fancyvrb}
\usepackage{newverbs}
\usepackage{listings}
\lstdefinestyle{TeXcode}{
  fancyvrb=true,
  language=[LaTeX]TeX,
  basicstyle=\ttfamily,
  keywordstyle=\color{blue}\bfseries,
  commentstyle=\color{red!50!black}\itshape,
  stringstyle=\ttfamily\color{green!50!black},
  showspaces=false,
  showstringspaces=false,
  backgroundcolor=\color{white},
  fontadjust=true,
  aboveskip=1ex,
  belowskip=1ex,
  emphstyle=\color{blue}\bfseries,
  keepspaces=true,
  flexiblecolumns=true,
  emph={switchOCG,showOCG,hideOCG,actionsOCG,usetikzlibrary}
}
\newenvironment{prog}[1]{\vspace{1ex}--- \texttt{\emph{#1}} ---}{}
\newcommand{\TeXexample}[3][]{
  \ifthenelse{\equal{#1}{}}{\begin{prog}{#2}}{\begin{prog}{#2 (#1)}}
  \lstinputlisting[style=TeXcode,aboveskip=0ex,#3]{#2}
  \end{prog}
}
  
\usepackage{url}
\usepackage{hyperref}
\usepackage[english]{babel}
\usepackage{parskip}
\usepackage{tikz}
\usetikzlibrary{shadows}
\usepackage[all]{nowidow}

\newcounter{mynotes}
\newenvironment{note}{%
  \tikzpicture
  \node[%draw=black,line width=.1pt,
  rounded corners=1em,
  inner xsep=1.2em, inner ysep=.7em,
  text width=\linewidth-2.4em,
  ,align=justify,
  fill=cyan!5,
  drop shadow,
  ] (note) \bgroup%
}{%
  \egroup;
  \node[circle,overlay,fill=white,draw,font=\small\bfseries,inner sep=2pt]
  at (note.west){\stepcounter{mynotes}\arabic{mynotes}};
  \endtikzpicture\par%
}

\newverbcommand\code{\color{red}}{}
\newverbcommand\style{\hspace{-2em}\color{red}}{}
\newcommand\argument[1]{\textcolor{black}{\ttfamily #1}}
\newcommand\TikZ{Ti\emph{k}Z}

\title{The \texttt{ocgx} package (version 0.1)}
\author{%
  Paul \textsc{Isambert}
  \and%
  Paul \textsc{Gaborit}\\\url{paul.gaborit@gmail.com}}
\date{\today}

\begin{document}
\maketitle

The \code+ocgx+ package extends and documents the \code+ocg+ package
(which comes with Asymptote) which allows you to create OCGs
(\emph{Optional Content Group}) in PDF documents.

Every OCG includes \TeX{} material into a layer of the PDF file. Each of
these layers can be displayed or not. Links can enable or disable the
display of OCGs.

The \code+ocgx+ package does not use Javascript embedded in the PDF
document to enable or disable OCGs. Thus, it is usable with several PDF
readers (to date, it has been successfully tested with \emph{Acrobat
  Reader}, \emph{Foxit Reader}, and \emph{Evince}).

\section{Usage}

Here is a simple example.

\TeXexample{ocgx-example-1.tex}{}

This document creates an OCG called \emph{ocg1} containing the text
``\emph{first example.}''  which is visible. You can show or hide this
OCG by clicking the link ``\emph{Button.}''.

\subsection{Create OCGs}

\begin{note}
  The \code+ocg+ package written by Michael \textsc{Ritzert}
  comes without documentation. In my knowledge, the only documentation
  for this package (by Kjell Magne Fauske) is offered on the weblog of
  TeXample.net:
  \url{http://www.texample.net/weblog/2008/nov/02/creating-pdf-layers/}.
\end{note}


\DescribeEnv{ocg}%
The \code+ocg+ environment (provided by the package \code+ocg+) creates
OCGs. It requires three arguments. The first argument is the name of the
OCG as it appears in the PDF viewer. The second argument is the internal
name used to reference this OCG. The third argument is a flag that
indicates whether the OCG should be visible or not (1 for visible, 0 for
invisible). The content of the environment (any \TeX material) is added
into the OCG.

The following code creates an OCG named \emph{OCG name} with
\emph{refocg} as internal reference. The content of this OCG is
``\emph{content...}''. This OCG is visible (the third argument is 1).

\begin{lstlisting}[style=TeXcode]
  \begin{ocg}{OCG name}{refocg}{1}
    content...
  \end{ocg}
\end{lstlisting}

\begin{note}
  A reference of an OCG consists of letters (A-Z, a-z), numbers (0-9)
  and possibly the \code+@+ character.
\end{note}

\begin{note}
  The content of the \code+ocg+ environment should not span across
  multiple pages. Currently, nothing prevents you to try it but the
  result will certainly not be the one you were expecting!
\end{note}

\begin{note}
  It is possible to nest an OCG in another OCG. To display the internal
  OCG, both the internal and external OCGs need to be in the visible
  state.
\end{note}

\begin{note}
  The same reference can be used with several \code+ocg+ environments (not
  necessarily in the same page). All materials are grouped in the same
  OCG. Only the first name provided will be used.
\end{note}

\subsection{Manage the visibility of OCGs}

\begin{note}
  Each link created by the following macros are placed in an
  \code+\hbox+. Therefore, it can not extend over several lines.
\end{note}

\DescribeMacro{\switchOCG}%
The \code+\switchOCG+ macro turns its second argument into a clickable
link that toggles the visibility status of all listed OCGs (by their
reference) in its first argument: if an OCG was visible, it becomes
invisible, and conversely, if an OCG was invisible, it becomes visible.

The following code creates the link \emph{toggle} that switches the
visibility status of OCGs whose references are \emph{ocg1} and
\emph{ocg2}:

\begin{lstlisting}[style=TeXcode]
  \switchOCG{ocg1 ocg2}{toggle}
\end{lstlisting}

\DescribeMacro{\showOCG}%
The \code+\showOCG+ macro turns its second argument into a clickable
link that make visible all OCGs whose references are listed in its first
argument: an invisible OCG becomes visible and an OCG already visible
remains visible.

The following code creates the link \emph{show} which makes visible the
OCGs whoses references are \emph{ocg1} and \emph{ocg2}:

\begin{lstlisting}[style=TeXcode]
  \showOCG{ocg1 ocg2}{show}
\end{lstlisting}

\DescribeMacro{\hideOCG}%
The \code+\hideOCG+ macro turns its second argument into a clickable
link that make invisible all OCGs whose references are listed in its
first argument: a visible OCG becomes invisible and an OCG already
invisible remains invisible.

The following code creates the link \emph{hide} which makes invisible
the OCGs whoses references are \emph{ocg1} and \emph{ocg2}:
\begin{lstlisting}[style=TeXcode]
  \hideOCG{ocg1 ocg2}{hide}
\end{lstlisting}

\DescribeMacro{\actionsOCG}%
The \code+\actionsOCG+ macro transforms its fourth argument in a
clickable link. Its three first arguments are lists of references of
OCGs. The first list contains references of OCGs which visibility status
is to be toggled. The second list contains references of OCGs to be set
visible. The third list contains references of OCGs to be set invisible.

The following code creates the link \emph{actions} to toggle the
visibility status of the OCG named \emph{ocg1}, to make visible the OCG
named \emph{ocg3}, and to make invisible OCG named \emph{ocg2}:

\begin{lstlisting}[style=TeXcode]
  \actionsOCG{ocg1}{ocg3}{ocg2}{actions}
\end{lstlisting}

\subsection{Usage with \TikZ{}}

You can use the \code+ocgx+ package with \TikZ{}. The package provides a
\TikZ{} library offering five specific styles to transform a path
(\code+\path+ or \code+\node+) in a clickable link. To use it, simply
add the following lines in your preamble:

\begin{lstlisting}[style=TeXcode]
  \usepackage{tikz}
  \usetikzlibrary{ocgx}
\end{lstlisting}

\begin{list}{}
\item
\noindent\style+/tikz/switch OCG+\argument{=\{\emph{<OCGs list>}\}}

This style transforms the path or the current node in a link acting as
if it was produced by the macro \code+\switchOCG+ (the visibility status
of referenced OCGs is reversed).

\noindent\style+/tikz/show OCG+\argument{=\{\emph{<OCGs list>}\}}

This style transforms the path or the current node in a link acting as
if it was produced by the macro \code+\showOCG+ (the referenced OCGs are
made visible).

\noindent\style+/tikz/hide OCG+\argument{=\{\emph{<OCGs list>}\}}

This style transforms the path or the current node in a link acting as
if it was produced by the macro \code+\hideOCG+ (the referenced OCGs are
made invisible).

\noindent\style+/tikz/actions OCG+\argument{=\{\emph{<OCGs
    list>}\}\{\emph{<OCGs list>}\}\{\emph{<OCGs list>}\}}

This style transforms the path or the current node in a link acting as
if it is produced by the macro \code+\actionsOCG+ (the visibility status
of OCGs of the first list is reversed, the OCGs in the second list are
made visible and those of the third list are made invisible).

\noindent\style+/tikz/switch OCG with mark+\argument{=\{\emph{<basename>}\}\{\emph{<OCGs list>}\}}

This style transforms the path or the current node in a link acting as
if it is produced by the macro \code+\switchOCG+ (the visibility status
of referenced OCGs in the list is reversed).

A mark (currently a simple cross) is drawn over the current path or node
in an OCG whose reference is \code+checkbox@basename+. The visibility
status of this new OCG will be reversed as those of the entire list.
\end{list}

\begin{note}
  Whatever the shape of the current path or node is, it is its
  \emph{bounding box} that is used to make the link (the link is always
  \emph{rectangular} and \emph{horizontal}).
\end{note}

\section{Examples}

The document \code+demo-ocgx.tex+ provides several examples of usage of
package \code+ocgx+ with \TikZ{} (and \code+beamer+).

\section{Limits and bugs}

\begin{enumerate}
\item Links are always horizontal rectangles!
\item An \code+ocg+ environment spanning across multiple pages are not
  detected and don't work correctly.
\item The list of OCGs created by \code+ocg+ is seen by viewers as a
  long flat list. Their possible hierarchy is not displayed.
\item The packages \code+ocg+ and \code+ocgx+ are not compatible with Plain-\TeX{}.
\end{enumerate}

\section{Development}
This package is still experimental. It is released on CTAN. You can
recover the latest version on \url{https://github.com/polgab/ocgx}. Any
help to participate in its development is welcome. Do not hesitate to
contact the maintener (\url{paul.gaborit@gmail.com}).

\end{document}



%%% Local Variables: 
%%% mode: latex
%%% TeX-master: t
%%% End: 
